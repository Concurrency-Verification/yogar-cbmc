\documentclass{article}

\newcommand{\dir}[1]{\texttt{#1}}
\newcommand{\file}[1]{\texttt{#1}}
\newcommand{\code}[1]{\texttt{#1}}
\newcommand{\prog}[1]{\texttt{#1}}

\title{Beginner's Guide to CPROVER}
\author{Martin Brain\thanks{But most of the content is from Michael Tautschnig}}

\begin{document}

\maketitle

\section{Background Information}

First off; read the CPROVER manual.  It describes how to get, build
and use CBMC and SATABS.  This document covers the internals of the
system and how to get started on development.


\subsection{Documentation}

Apart from the (user-orientated) CPROVER manual and this document,
most of the rest of the documentation is inline in the code
as \texttt{doxygen} and some comments.  A man page for CBMC, goto-cc
and goto-instrument is contained in the \dir{doc/} directory and gives
some options for these tools.  All of these could be improved
and patches are very welcome.  In some cases the algorithms used are
described in the relevant papers.

\subsection{Architecture}

CPROVER is structured in a similar fashion to a compiler.  It has
language specific front-ends which perform limited syntactic analysis
and then convert to an intermediate format.  The intermediate format
can be output to files (this is what \texttt{goto-cc} does) and are
(informally) referred to as ``goto binaries'' or ``goto programs''.
The back-end are tools process this format, either directly from the
front-end or from it's saved output.  These include a wide range of
analysis and transformation tools (see Section \ref{section:other-apps}).

\subsection{Coding Standards}

CPROVER is written in a fairly minimalist subset of C++; templates and
meta-programming are avoided except where necessary.  The standard
library is used but in many cases there are alternatives provided in
\dir{util/} (see Section \ref{section:util}) which are preferred.
Boost is not used.

Patches should be formatted so that code is indented with two space
characters, not tab and wrapped to 75 or 72 columns.  Headers for
doxygen should be given (and preferably filled!) and the author will
be the person who first created the file.

Identifiers should be lower case with underscores to separate words.
Types (classes, structures and typedefs) names must\footnote{There are
a couple of exceptions, including the graph classes} end with a
\code{t}.  Types that model types (i.e. C types in the program that is
being interpreted) are named with \code{\_typet}.
For example \code{ui\_message\_handlert} rather than
\code{UI\_message\_handlert} or \code{UIMessageHandler} and
\code{union\_typet}.



\subsection{How to Contribute}

Fixes, changes and enhancements to the CPROVER code base should be
developed against the \texttt{trunk} version and submitted to Daniel
as patches produced by \texttt{diff -Naur} or \texttt{svn diff}.
Entire applications are best developed independently (\texttt{git svn}
is a popular choice for tracking the main trunk but also having local
development) until it is clear what their utility, future and
maintenance is likely to be.


\subsection{Other Useful Code}
\label{section:other-apps}

The CPROVER subversion archive contains a number of separate
programs.  Others are developed separately as patches or separate
branches.%  New applications are initially developed in their version
%control system and may be merged into the main subversion system
%depending on their utility, popularity and maintenance.
Interfaces are have been and are continuing to stablise but older code
may require work to compile and function correctly.

In the main archive:

\begin{description}
  \item[\prog{CBMC}]{A bounded model checking tool for C and C++. See
    Section \ref{section:CBMC}.}
  \item[\prog{goto-cc}]{A drop-in, flag compatible replacement for GCC
    and other compilers that produces goto-programs rather than
    executable binaries.  See Section \ref{section:goto-cc}.}
  \item[\prog{goto-instrument}]{A collection of functions for
    instrumenting and modifying goto-programs.  See Section
    \ref{section:goto-instrument}.}
\end{description}

Model checkers and similar tools:

\begin{description}
  \item[\prog{SatABS}]{A CEGAR model checker using predicate
    abstraction.  Is roughly 10,000 lines of code (on top of the CPROVER
    code base) and is developed in its own subversion archive.  It
    uses an external model checker to find potentially feasible paths.
    Key limitations are related to code with pointers and there is
    scope for significant improvement.}

  \item[\prog{Scratch}]{Alistair Donaldson's k-induction based tool.
    The front-end is in the old project CVS and some of the
    functionality is in \prog{goto-instrument}.}

  \item[\prog{Wolverine}]{An implementation of Ken McMillan's IMPACT
    algorithm for sequential programs.  In the old project CVS.}

  \item[\prog{C-Impact}]{An implementation of Ken McMillan's IMPACT
    algorithm for parallel programs.  In the old project CVS.}

  \item[\prog{LoopFrog}]{A loop summarisation tool.}

  \item[\prog{???}]{Christoph's termination analyser.}

\end{description}


Test case generation:

\begin{description}
  \item[\prog{cover}]{A basic test-input generation tool.  In the old
    project CVS.}

  \item[\prog{FShell}]{A test-input generation tool that allows the
      user to specify the desired coverage using a custom language
      (which includes regular expressions over paths).  It uses
      incremental SAT and is thus faster than the na\"ive ``add
      assertions one at a time and use the counter-examples''
      approach.  Is developed in its own subversion.}
\end{description}



Alternative front-ends and input translators:

\begin{description}
  \item[\prog{Scoot}]{A System-C to C translator.  Probably in the old
    project CVS.}
  
  \item[\prog{???}]{A Simulink to C translator.  In the old project CVS.}

  \item[\prog{???}]{A Verilog front-end.  In the old project CVS.}

  \item[\prog{???}]{A converter from Codewarrior project files to
    Makefiles.  In the old project CVS.}
\end{description}


Other tools:

\begin{description}
  \item[\prog{ai}]{Leo's hybrid abstract interpretation / CEGAR tool.}

  \item[\prog{DeltaCheck?}]{Ajitha's slicing tool, aimed at locating
    changes and differential verification.  In the old project CVS.}
\end{description}


There are tools based on the CPROVER framework from other research
groups which are not listed here.






\section{Source Walkthrough}

This section walks through the code bases in a rough order of interest
/ comprehensibility to the new developer.



\subsection{\dir{doc}}

At the moment just contains the CBMC man page.

\subsection{\dir{regression/}}

The regression tests are currently being moved from CVS.  The
\dir{regression/} directory contains all of those that have been
moved.  They are grouped into directories for each of the tools.  Each
of these contains a directory per test case, containing a C or C++
file that triggers the bug and a \file{.dsc} file that describes the
tests, expected output and so on.  There is a Perl script,
\file{test.pl} that is used to invoke the tests as:

\begin{center}
  \code{../test.pl -c PATH\_TO\_CBMC}
\end{center}

The \code{--help} option gives instructions for use and the format of
the description files.



\subsection{\dir{src/}}

The source code is divided into a number of sub-directories, each
containing the code for a different part of the system.  In the top
level files there are only a few files:

\begin{description}
  \item[\file{config.inc}]{The user-editable configuration parameters
    for the build process.  The main use of this file is setting the
    paths for the various external SAT solvers that are used.  As
    such, anyone building from source will likely need to edit this.}
  \item[\file{Makefile}]{The main systems Make file.  Parallel builds
    are supported and encouraged; please don't break them!}
  \item[\file{common}]{System specific magic required to get the
    system to build.  This should only need to be edited if porting
    CBMC to a new platform / build environment.}
  \item[\file{doxygen.cfg}]{The config file for doxygen.cfg}
\end{description}



\subsubsection{\dir{util/}}
\label{section:util}

\dir{util/} contains the low-level data structures and manipulation
functions that are used through-out the CPROVER code-base.  For almost
any low-level task, the code required is probably in \dir{util/}.  Key
files include:

\begin{description}
  \item[\file{irep.h}]{This contains the definition of \code{irept},
    the basis of many of the data structures in the project.  They
    should not be used directly; one of the derived classes should be
    used.  For more information see Section \ref{section:irept}.}
  \item[\file{expr.h}]{The parent class for all of the expressions.
    Provides a number of generic functions, \code{exprt} can be used
    with these but when creating data, subclasses of \code{exprt}
    should be used.}
  \item[\file{std\_expr.h}]{Provides subclasses of \code{exprt} for
    common kinds of expression for example \code{plus\_exprt},
    \code{minus\_exprt}, \code{dereference\_exprt}.  These are the
    intended interface for creating expressions.}
  \item[\file{std\_types.h}]{Provides subclasses of \code{typet} (a
    subclass of \code{irept}) to model C and C++ types.  This is one
    of the preferred interfaces to \code{irept}.  The front-ends handle
    type promotion and most coercision so the type system and checking
    goto-programs is simpler than C.}
  \item[\file{dstring.h}]{The CPROVER string class.  This enables
    sharing between strings which significantly reduces the amount of
    memory required and speeds comparison.  \code{dstring} should not
    be used directly, \code{irep\_idt} should be used instead, which
    (dependent on build options) is an alias for \code{dstring}.}
  \item[\file{mp\_arith.h}]{The wrapper class for multi-precision
    arithmetic within CPROVER.  Also see \file{arith\_tools.h}.}
  \item[\file{ieee\_float.h}]{The arbitrary precision float model used
    within CPROVER.  Based on \code{mp\_integer}s.}
  \item[\file{context.h}]{A generic container for symbol table like
    constructs such as namespaces.  Lookup gives type, location of
    declaration, name, `pretty name', whether it is static or not.}
  \item[\file{namespace.h}]{The preferred interface for the context
    class.  The key function is \code{lookup} which converts a string
    (\code{irep\_idt}) to a symbol which gives the scope of
    declaration, type and so on.  This works for functions as well as variables.}
\end{description}




\subsubsection{\dir{langapi/}}

This contains the basic interfaces and support classes for programming
language front ends.  Developers only really need look at this if they
are adding support for a new language.  It's main users are the two
(in trunk) language front-ends; \dir{ansi-c/} and \dir{cpp/}.


\subsubsection{\dir{ansi-c/}}

Contains the front-end for ANSI C, plus a variety of common
extensions.  This parses the file, performs some basic sanity checks
(this is one area in which the UI could be improved; patches most
welcome) and then produces a goto-program (see below).  The parser is
a traditional Flex / Bison system.  

\file{internal\_addition.c} contains the implementation of various
`magic' functions that are that allow control of the analysis from the
source code level.  These include assertions, assumptions, atomic
blocks, memory fences and rounding modes.

The \dir{library/} subdirectory contains versions of some of the C
standard header files that make use of the CPROVER built-in
functions.  This allows CPROVER programs to be `aware' of the
functionality and model it correctly.  Examples include
\file{stdio.c}, \file{string.c}, \file{setjmp.c} and various threading
interfaces.


\subsubsection{\dir{cpp/}}

This directory contains the C++ front-end.  It supports the subset of
C++ commonly found in embedded and system applications.
Consequentially it doesn't have full support for templates and many of
the more advanced and obscure C++ features.  The subset of the
language that can be handled is being extended over time so bug
reports of programs that cannot be parsed are useful.

The functionality is very similar to the ANSI C front end; parsing the
code and converting to goto-programs.  It makes use of code from
\dir{langapi} and \dir{ansi-c}.




\subsubsection{\dir{goto-programs/}}

Goto programs are the intermediate representation of the CPROVER tool
chain.  They are language independent and similar to many of the
compiler intermediate languages.  Section \ref{section:goto-programs}
describes the \code{goto\_programt} and \code{goto\_functionst} data
structures in detail.  However it useful to understand some of the
basic concepts.  Each function is a list of instructions, each of
which has a type (one of 18 kinds of instruction), a code expression,
a guard expression and potentially some targets for the next
instruction.  They are not natively in static single-assign (SSA)
form.  Transitions are nondeterministic (although in practise the
guards on the transitions normally cover form a disjoint cover of all
possibilities).  Local variables have non-deterministic values if they
are not initialised.  Variables and data within the program is
commonly one of three types (parameterised by width):
\code{unsignedbv\_typet}, \code{signedbv\_typet} and
\code{floatbv\_typet}, see \file{util/std\_types.h} for more
information. Goto programs can be serialised in a binary (wrapped in
ELF headers) format or in XML (see the various \code{\_serialization}
files).


The \prog{cbmc} option \code{--show-goto-programs} is often a
good starting point as it outputs goto-programs in a human
readable form.  However there are a few things to be aware of.
Functions have an internal name (for example \code{c::f00}) and a
`pretty name' (for example \code{f00}) and which is used depends on
whether it is internal or being presented to the user.  The
\code{main} method is the `logical' main which is not necessarily the
main method from the code.  In the output \code{NONDET} is use to
represent a nondeterministic assignment to a variable.  Likewise
\code{IF} as a beautified \code{GOTO} instruction where the guard
expression is used as the condition.  \code{RETURN} instructions may
be dropped if they precede an \code{END\_FUNCTION} instruction.  The
comment lines are generated from the \code{locationt} field of the
\code{instructiont} structure.

\dir{goto-programs/} is one of the few places in the CPROVER codebase
that templates are used.  The intention is to allow the general
architecture of program and functions to be used for other
formalisms.  At the moment most of the templates have a single
instantiation; for example \code{goto\_functionst} and
\code{goto\_function\_templatet} and \code{goto\_programt} and \code{goto\_program\_templatet}.



\subsubsection{\dir{goto-symex/}}

This directory contains a symbolic evaluation system for
goto-programs.  This takes a goto-program and translates it to an
equation system by traversing the program, branching and merging and
unwinding loops as needed.  Each reverse goto has a separate counter
(the actual counting is handled by \prog{cbmc}, see the \code{--unwind}
and \code{--unwind-set} options).  When a counter limit
is reach, an assertion can be added to explicitly show when analysis
is incomplete.  The symbolic execution includes constant folding so
loops that have a constant number of iterations will be handled
completely (assuming the unwinding limit is sufficient).

The output of the symbolic execution is a system of equations; an
object containing a list of \code{symex\_target\_elements}, each of
which are equalities between \prog{expr} expressions.  See
\file{symex\_target\_equation.h}.  The output is in static, single
assignment (SSA) form, which is \emph{not} the case for goto-programs.



\subsubsection{\dir{pointer-analysis/}}

To perform symbolic execution on programs with dereferencing of
arbitrary pointers, some alias analysis is needed.
\dir{pointer-analysis} contains the three levels of analysis; flow and
context insensitive, context sensitive and flow and context
sensitive.  The code needed is subtle and sophisticated and thus there
may be bugs.




\subsubsection{\dir{solvers/}}

The \dir{solvers/} directory contains interfaces to a number of
different decision procedures, roughly one per directory.  

\begin{description}

  \item[prop/]{The basic and common functionality.  The key file is
    \file{prop\_conv.h} which defines \code{prop\_convt}.  This is the
     base class that is used to interface to the decision procedures.
     The key functions are \code{convert} which takes an \code{exprt}
     and converts it to the appropriate, solver specific, data
     structures and \code{dec\_solve} (inherited from
     \code{decision\_proceduret}) which invokes the actual decision
     procedures.  Individual decision procedures (named
     \code{*\_dect}) objects can be created but \code{prop\_convt} is
     the preferred interface for code that uses them.}  

  \item[flattening/]{A library that converts operations to
    bit-vectors, including calling the conversions in \dir{floatbv} as
    necessary.  Is implemented as a simple conversion (with caching)
    and then a post-processing function that adds extra constraints.
    This is not used by the SMT or CVC back-ends.}  

  %%%%

  \item[dplib/]{Provides the \code{dplib\_dect} object which used the
    decision procedure library from ``Decision Procedures : An
    Algorithmic Point of View''.}  

  \item[cvc/]{Provides the \code{cvc\_dect} type which interfaces to
    the old (pre SMTLib) input format for the CVC family of solvers.
    This format is still supported by depreciated in favour of SMTLib 2.}  

  \item[smt1/]{Provides the \code{smt1\_dect} type which converts the
    formulae to SMTLib version 1 and then invokes one of Boolector,
    CVC3, OpenSMT, Yices, MathSAT or Z3.  Again, note that this format
    is depreciated.}  

  \item[smt2/]{Provides the \code{smt2\_dect} type which functions in
    a similar way to \code{smt1\_dect}, calling Boolector, CVC3,
    MathSAT, Yices or Z3.  Note that the interaction with the solver
    is batched and uses temporary files rather than using the
    interactive command supported by SMTLib 2.  With the \code{--fpa}
    option, this output mode will not flatten the floating point
    arithmetic and instead output the proposed SMTLib floating point
    standard.}
  
  \item[qbf/]{Back-ends for a variety of QBF solvers.  Appears to be
    no longer used or maintained.}
  
  \item[sat/]{Back-ends for a variety of SAT solvers and DIMACS
    output.}  
\end{description}





\subsubsection{\dir{cbmc/}}
\label{section:CBMC}

This contains the first full application.  CBMC is a bounded model
checker that uses the front ends (\dir{ansi-c}, \dir{cpp}, goto-program or others)
to create a goto-program, \dir{goto-symex} to unwind the loops the given
number of times and to produce and equation system and finally
\dir{solvers} to find a counter-example (technically, \dir{goto-symex}
is then used to construct the counter-example trace).



\subsubsection{\dir{goto-cc/}}
\label{section:goto-cc}

\dir{goto-cc} is a compiler replacement that just performs the first
step of the process; converting C or C++ programs to goto-binaries.
It is intended to be dropped in to an existing build procedure in
place of the compiler, thus it emulates flags that would affect the
semantics of the code produced.  Which set of flags are emulated
depends on the naming of the \dir{goto-cc/} binary.  If it is called
\prog{goto-cc} then it emulates GCC flags, \prog{goto-armcc} emulates
the ARM compiler, \prog{goto-cl} emulates VCC and \prog{goto-cw}
emulates the Code Warrior compiler.  The output of this tool can then
be used with \prog{cbmc} or \prog{goto-instrument}.




\subsubsection{\dir{goto-instrument/}}
\label{section:goto-instrument}

The \dir{goto-instrument/} directory contains a number of tools, one
per file, that are built into the \prog{goto-instrument} program.  All
of them take in a goto-program (produced by \prog{goto-cc}) and either
modify it or perform some analysis.  Examples include
\file{nondet\_static.cpp} which initialises static variables to a
non-deterministic value, \file{nondet\_volatile.cpp} which assigns
a non-deterministic value to any volatile variable before it is read
and \file{weak\_memory.h} which performs the necessary transformations
to reason about weak memory models.  The exception to the ``one file
for each piece of functionality'' rule are the program instrumentation
options (mostly those given as ``Safety checks'' in the
\prog{goto-instrument} help text) which are included in the
\prog{goto-program/} directory.  An example of this is
\file{goto-program/stack\_depth.h} and the general rule seems to be
that transformations and instrumentation that \prog{cbmc} uses should
be in \dir{goto-program/}, others should be in \dir{goto-instrument}.

\prog{goto-instrument} is a very good template for new analysis
tools.  New developers are advised to copy the directory, remove all
files apart from \file{main.*}, \file{parseoptions.*} and the
\file{Makefile} and use these as the skeleton of their application.
The \code{doit()} method in \file{parseoptions.cpp} is the preferred
location for the top level control for the program.



\subsubsection{\dir{linking/}}

Probably the code to emulate a linker.  This allows multiple `object
files' (goto-programs) to be linked into one `executable' (another
goto-program), thus allowing existing build systems to be used to
build complete goto-program binaries.


\subsubsection{\dir{big-int/}}

CPROVER is distributed with its own multi-precision arithmetic
library; mainly for historical and portability reasons.  The library is externally
developed and thus \dir{big-int} contains the source as it is
distributed.  This should not be used directly, see
\file{util/mp\_arith.h} for the CPROVER interface.



\subsubsection{\dir{xmllang/}}

CPROVER has optional XML output for results and there is an XML format
for goto-programs.  It is used to interface to various IDEs.  The
\dir{xmllang/} directory contains the parser and helper functions for
handling this format.



\subsubsection{\dir{floatbv/}}

This library contains the code that is used to convert floating point
variables (\code{floatbv}) to bit vectors (\code{bv}).  This is
referred to as `bit-blasting' and is called in the \dir{solver} code
during conversion to SAT or SMT.  It also contains the abstraction
code described in the FMCAD09 paper.








\section{Data Structures}

This section discusses some of the key data-structures used in the
CPROVER codebase.

\subsection{\code{irept}}
\label{section:irept}

There are a large number of kind of tree structured or tree-like data
in CPROVER.  \code{irept} provides a single, unified representation for
all of these, allowing structure sharing and reference counting of
data.  As such \code{irept} is the basic unit of data in CPROVER.
Each \code{irept} contains\footnote{Or references, if reference
  counted data sharing is enabled.  It is enabled by default; see the
  \code{SHARING} macro.} a basic unit of data (of type \code{dt})
which contains four things:

\begin{description}
\item[\code{data}]{A string\footnote{When \code{USE\_DSTRING} is enabled (it
    is by default), this is actually a \code{dstring} and thus an
    integer which is a reference into a string table}, which is
    returned when the \code{id()} function is used.}
\item[\code{named\_sub}]{A map from \code{irep\_namet} (a string) to
    an \code{irept}.  This is used for named children,
    i.e. subexpressions, parameters, etc.}
\item[\code{comments}]{Another map from \code{irep\_namet} to
  \code{irept} which is used for annotations and other `non-semantic' information}
\item[\code{sub}]{A vector of \code{irept} which is used to store
  ordered but unnamed children.}
\end{description}

The \code{irept::pretty} function outputs the contents of an
\code{irept} directly and can be used to understand an debug problems
with \code{irept}s.


On their own \code{irept}s do not ``mean'' anything; they are
effectively generic tree nodes.  Their interpretation depends on the
contents of result of the \code{id} function (the \code{data}) field.
\file{util/irep\_ids.txt} contains the complete list of \code{id}
values.  During the build process it is used to generate
\file{util/irep\_ids.h} which gives constants for each id (named
\code{ID\_*}).  These can then be used to identify what kind of data
\code{irept} stores and thus what can be done with it.

To simplify this process, there are a variety of classes that inherit
from \code{irept}, roughly corresponding to the ids listed
(i.e. \code{ID\_or} (the string \code{"or''}) corresponds to the class
\code{or\_exprt}).  These give semantically relevant accessor
functions for the data; effectively different APIs for the same
underlying data structure.  None of these classes add fields (only
methods) and so static casting can be used.  The inheritance graph of
the subclasses of \code{irept} is a useful starting point for working
out how to manipulate data.

There are three main groups of classes (or APIs); those derived from
\code{typet}, \code{codet} and \code{exprt} respectively.  Although
all of these inherit from \code{irept}, these are the most abstract
level that code should handle data.  If code is manipulating plain
\code{irept}s then something is wrong with the architecture of the
code.

Many of the key descendent of \code{exprt} are declared in
\file{std\_expr.h}.  All expressions have a named subfield /
annotation which gives the type of the expression (slightly
simplified from C/C++ as \code{unsignedbv\_typet},
\code{signedbv\_typet}, \code{floatbv\_typet}, etc.).  All type
conversions are explicit with an expression with \code{id() ==
  ID\_typecast} and an `interface class' named
\code{typecast\_exprt}.  One key descendent of \code{exprt} is
\code{symbol\_exprt} which creates \code{irept} instances with the id
of ``symbol''.  These are used to represent variables; the name of
which can be found using the \code{get\_identifier} accessor function.


\code{codet} inherits from \code{exprt} and is defined in
\file{std\_code.h}.  They represent executable code; statements in C
rather than expressions.  In the front-end there are versions of these
that hold whole code blocks, but in goto-programs these have been
flattened so that each \code{irept} represents one sequence point
(almost one line of code / one semi-colon).  The most common
descendents of \code{codet} are \code{code\_assignt} so a common
pattern is to cast the \code{codet} to an assignment and then recurse
on the expression on either side.






\subsection{\code{goto-programs}}
\label{section:goto-programs}

The common starting point for working with goto-programs is the
\code{read\_goto\_binary} function which populates an object of
\code{goto\_functionst} type.  This is defined in
\file{goto\_functions.h} and is an instantiation of the template
\code{goto\_functions\_templatet} which is contained in
\file{goto\_functions\_template.h}.  They are wrappers around a map from
strings to \code{goto\_programt}'s and iteration macros are provided.
Note that \code{goto\_function\_templatet} (no \code{s}) is defined in
the same header as \code{goto\_functions\_templatet} and is gives the
C type for the function and Boolean which indicates whether the body
is available (before linking this might not always be true).  Also
note the slightly counter-intuitive naming; \code{goto\_functionst}
instances are the top level structure representing the program and
contain \code{goto\_programt} instances which represent the individual
functions.  At the time of writing \code{goto\_functionst} is the only
instantiation of the template \code{goto\_functions\_templatet} but
other could be produced if a different data-structures / kinds of models
were needed for functions.

\code{goto\_programt} is also an instantiation of a template.  In a
similar fashion it is \code{goto\_program\_templatet} and allows the
types of the guard and expression used in instructions to be
parameterised.  Again, this is currently the only use of the template.
As such there are only really helper functions in
\file{goto\_program.h} and thus \code{goto\_program\_template.h} is
probably the key file that describes the representation of (C)
functions in the goto-program format.  It is reasonably stable and
reasonably documented and thus is a good place to start looking at the code.

An instance of \code{goto\_program\_templatet} is effectively a list
of instructions (and inner template called \code{instructiont}).  It
is important to use the copy and insertion functions that are provided
as iterators are used to link instructions to their predecessors and
targets and careless manipulation of the list could break these.
Likewise there are helper macros for iterating over the instructions
in an instance of \code{goto\_program\_templatet} and the use of these
is good style and strongly encouraged.

Individual instructions are instances of type \code{instructiont}.
They represent one step in the function.  Each has a type, an instance
of \code{goto\_program\_instruction\_typet} which denotes what kind of
instruction it is.  They can be computational (such as \code{ASSIGN}
or \code{FUNCTION\_CALL}), logical (such as \code{ASSUME} and
\code{ASSERT}) or informational (such as \code{LOCATION} and
\code{DEAD}).  At the time of writing there are 18 possible values for
\code{goto\_program\_instruction\_typet} / kinds of instruction.
Instructions also have a guard field (the condition under which it is
executed) and a code field (what the instruction does).  These may be
empty depending on the kind of instruction.  In the default
instantiations these are of type \code{exprt} and \code{codet}
respectively and thus covered by the previous discussion of
\code{irept} and its descendents.  The next instructions (remembering
that transitions are guarded by non-deterministic) are given by the
list \code{targets} (with the corresponding list of labels
\code{labels}) and the corresponding set of previous instructions is
get by \code{incoming\_edges}.  Finally \code{instructiont} have
informational \code{function} and \code{location} fields that indicate
where they are in the code.



\end{document}
